% !TeX root = ../main.tex

% 中英文摘要和关键字

\begin{abstract}
  论文摘要包括中文摘要和英文摘要(Abstract)两部分。摘要是论文内容的简要陈述,是一篇具有独立性和完整性的短文,
  应概括地反映出本论文的主要内容,说明本论文的主要研究目的、内容、方法、成果和结论。要突出本论文的创造性成果或新见解,
  不宜使用公式、图表、表格或其他插图材料,不标注引用文献。中文摘要力求语言精炼准确,博士论文一般约为1000字(word统计),
  硕士论文一般约为600字(word统计)。英文摘要与中文摘要内
  摘要正文内容一般包括:从事这项研究工作的目的和意义;作者独立进行的研究工作的概括性叙述;研究获得的主要结论或提出的主要观点。
  硕士学位论文摘要应突出论文的新见解,博士学位论文摘要应突出论文的创新点。
  
  关键词在摘要正文内容后另起一行标明,一般3~5个,之间用分号分开, 最后一个关键词后不打标点符号。关键词是为了文献索引和检索工作,
  从论文中选取出来,用以表示全文主题内容信息的单词或术语,应体现论文特色,具有语义性,在论文中有明确出处。
  应尽量采用《汉语主题词表》或各专业主题词表提供的规范词。
  摘要页应单独编页。
  % 关键词用“英文逗号”分隔
  \scusetup{
    keywords = {LaTeX, CJK, 模板, 四川大学},
  }
\end{abstract}

\begin{abstract*}
  论文摘要包括中文摘要和英文摘要(Abstract)两部分。摘要是论文内容的简要陈述,是一篇具有独立性和完整性的短文,
  应概括地反映出本论文的主要内容,说明本论文的主要研究目的、内容、方法、成果和结论。要突出本论文的创造性成果或新见解,
  不宜使用公式、图表、表格或其他插图材料,不标注引用文献。中文摘要力求语言精炼准确,博士论文一般约为1000字(word统计),
  硕士论文一般约为600字(word统计)。英文摘要与中文摘要内
  摘要正文内容一般包括:从事这项研究工作的目的和意义;作者独立进行的研究工作的概括性叙述;研究获得的主要结论或提出的主要观点。
  硕士学位论文摘要应突出论文的新见解,博士学位论文摘要应突出论文的创新点。
  
  关键词在摘要正文内容后另起一行标明,一般3~5个,之间用分号分开, 最后一个关键词后不打标点符号。关键词是为了文献索引和检索工作,
  从论文中选取出来,用以表示全文主题内容信息的单词或术语,应体现论文特色,具有语义性,在论文中有明确出处。
  应尽量采用《汉语主题词表》或各专业主题词表提供的规范词。

  摘要页应单独编页。
  \scusetup{
    keywords* = {LaTeX, CJK, template, scu},
  }
\end{abstract*}
