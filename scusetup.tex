% !TeX root = ./main.tex

% 论文基本信息配置

\scusetup{
  %******************************
  % 注意:
  %   1. 配置里面不要出现空行
  %   2. 不需要的配置信息可以删除
  %******************************
  %
  % 标题
  %   不可使用“\\”命令手动控制换行,一行写不下时可分两行写。分两行写时要手动将标题匀到 sub-title 中
  %
  title  = {党所贵的学位论文},
  title* = {Dang's Thesis},
  % sub-title  = {fgf dsdg d fgdsg },
  %
  % 学位
  % 学科名称以国务院学位委员会颁布的《授予博士、硕士学位和培养研究生的学科、专业目录》为准,
  %     1. 学术学位研究生的学位类别填写为:理(工、农)学博士或理(工、农)学硕士,
  %        学科专业填写为:一级学科-二级学科,如:生物学-动物学,生态学-动物生态学。
  %     2.专业学位研究生的学位类别填写为:林业硕士或工程硕士,
  %        领域名称填写:生物医学工程或生物工程,林业硕士未分设领域,填写:无。
  %
  degree-name  = {工学硕士},
  degree-name* = {Master of Engineering},
  %
  % 培养单位
  %   填写所属院系的全名
  %
  department = {计算机学院},
  %
  % 学科名称
  %   以国务院学位委员会颁布的《授予博士、硕士学位和培养研究生的学科、专业目录》为准,
  %     1. 学术学位研究生的学位类别填写为:理(工、农)学博士或理(工、农)学硕士,
  %        学科专业填写为:一级学科-二级学科,如:生物学-动物学,生态学-动物生态学。
  %     2.专业学位研究生的学位类别填写为:林业硕士或工程硕士,
  %        领域名称填写:生物医学工程或生物工程,林业硕士未分设领域,填写:无。
  %
  discipline  = {计算机科学与技术},
  discipline* = {Computer Science and Technology},
  %
  % 姓名
  %
  author  = {党所贵},
  author* = {Dang Suogui},
  %
  % 指导教师
  %   填写经培养单位批准备案的的导师姓名,并加上专业技术职称(联合培养专项计划博士研究生填写双导师信息)。
  %   中文姓名和职称之间以英文逗号“,”分开
  %
  supervisor  = {燕锐,教授},
  supervisor* = {Prof., Yan Rui},
  %
  % 副指导教师
  %   填写经培养单位批准备案的的导师姓名,并加上专业技术职称(联合培养专项计划博士研究生填写双导师信息)。
  %   中文姓名和职称之间以英文逗号“,”分开
  %
  %associate-supervisor  = {燕锐,教授},
  %associate-supervisor* = {Prof., Yan Rui},
  %
  % 联合指导教师
  %
  % joint-supervisor  = {燕锐,教授},
  % joint-supervisor* = {Prof., Yan Rui},
  %
  % 论文完成日期
  %   使用 ISO 格式;默认为当前时间
  %
  % date = {2020-02-02},
  %
  % 论文答辩日期
  %   使用 ISO 格式;默认为当前时间
  %
  % presentdate = {2020-02-02},
  %
  % 学位授予日期
  %   使用 ISO 格式;默认为当前时间
  %
  % degreedate = {2020-02-02},
  %
  % 密级和年限
  %   涉密论文必须在论文封面标注密级,同时注明保密年限。公开论文不标注密级,可删除。
  %   没有公布具体的格式样式,此处先占位,有需要再实现。
  %
  % secret-level = {秘密},
  % secret-year  = {10},
  %
  %
  % 学号。
  %  stuid = {1234},
  %
  % 送审编号。
  % 
  %  reviewid = {1234},
  %
  % 单位编号。
  %    默认是 10610
  %    departmentid = {1254},
}

%% 下面放需要用到pkg,有额外需要的可以在这里添加

% 表格中支持跨行
\usepackage{multirow}

% 跨页表格
\usepackage{longtable}

% 固定宽度的表格
\usepackage{tabularx}

% 表格中的反斜线
\usepackage{diagbox}

% 确定浮动对象的位置,可以使用 H,强制将浮动对象放到这里
\usepackage{float}

% 浮动图形控制宏包。
% 允许上一个 section 的浮动图形出现在下一个 section 的开始部分
% 该宏包提供处理浮动对象的 \FloatBarrier 命令,使所有未处
% 理的浮动图形立即被处理。这三个宏包仅供参考,未必使用:
% \usepackage[below]{placeins}
% \usepackage{floatflt} % 图文混排用宏包
% \usepackage{rotating} % 图形和表格的控制旋转

% 定理类环境宏包
\usepackage[amsmath,thmmarks,hyperref]{ntheorem}

% 给自定义的宏后面自动加空白
% \usepackage{xspace}

% 借用 ltxdoc 里面的几个命令。
\def\cmd#1{\cs{\expandafter\cmd@to@cs\string#1}}
\def\cmd@to@cs#1#2{\char\number`#2\relax}
\DeclareRobustCommand\cs[1]{\texttt{\char`\\#1}}

\newcommand*{\meta}[1]{{%
  \ensuremath{\langle}\rmfamily\itshape#1\/\ensuremath{\rangle}}}
\providecommand\marg[1]{%
  {\ttfamily\char`\{}\meta{#1}{\ttfamily\char`\}}}
\providecommand\oarg[1]{%
  {\ttfamily[}\meta{#1}{\ttfamily]}}
\providecommand\parg[1]{%
  {\ttfamily(}\meta{#1}{\ttfamily)}}
\providecommand\pkg[1]{{\sffamily#1}}

% 定义所有的图片文件在 figures 子目录下
\graphicspath{{figures/}}

% 数学命令,按照自己的习惯,添加修改。
\input{math_commands.tex}

% 定义自己常用的东西,例如
% Figure reference, lower-case.
% \def\figref#1{figure~\ref{#1}}
% Figure reference, capital. For start of sentence
% \def\Figref#1{Figure~\ref{#1}}

% hyperref 宏包在最后调用
\usepackage{hyperref}
